\documentclass[]{article}
\usepackage{lmodern}
\usepackage{amssymb,amsmath}
\usepackage{ifxetex,ifluatex}
\usepackage{fixltx2e} % provides \textsubscript
\ifnum 0\ifxetex 1\fi\ifluatex 1\fi=0 % if pdftex
  \usepackage[T1]{fontenc}
  \usepackage[utf8]{inputenc}
\else % if luatex or xelatex
  \ifxetex
    \usepackage{mathspec}
  \else
    \usepackage{fontspec}
  \fi
  \defaultfontfeatures{Ligatures=TeX,Scale=MatchLowercase}
\fi
% use upquote if available, for straight quotes in verbatim environments
\IfFileExists{upquote.sty}{\usepackage{upquote}}{}
% use microtype if available
\IfFileExists{microtype.sty}{%
\usepackage{microtype}
\UseMicrotypeSet[protrusion]{basicmath} % disable protrusion for tt fonts
}{}
\usepackage[margin=1in]{geometry}
\usepackage{hyperref}
\hypersetup{unicode=true,
            pdftitle={NB},
            pdfauthor={MENG},
            pdfborder={0 0 0},
            breaklinks=true}
\urlstyle{same}  % don't use monospace font for urls
\usepackage{color}
\usepackage{fancyvrb}
\newcommand{\VerbBar}{|}
\newcommand{\VERB}{\Verb[commandchars=\\\{\}]}
\DefineVerbatimEnvironment{Highlighting}{Verbatim}{commandchars=\\\{\}}
% Add ',fontsize=\small' for more characters per line
\usepackage{framed}
\definecolor{shadecolor}{RGB}{248,248,248}
\newenvironment{Shaded}{\begin{snugshade}}{\end{snugshade}}
\newcommand{\KeywordTok}[1]{\textcolor[rgb]{0.13,0.29,0.53}{\textbf{#1}}}
\newcommand{\DataTypeTok}[1]{\textcolor[rgb]{0.13,0.29,0.53}{#1}}
\newcommand{\DecValTok}[1]{\textcolor[rgb]{0.00,0.00,0.81}{#1}}
\newcommand{\BaseNTok}[1]{\textcolor[rgb]{0.00,0.00,0.81}{#1}}
\newcommand{\FloatTok}[1]{\textcolor[rgb]{0.00,0.00,0.81}{#1}}
\newcommand{\ConstantTok}[1]{\textcolor[rgb]{0.00,0.00,0.00}{#1}}
\newcommand{\CharTok}[1]{\textcolor[rgb]{0.31,0.60,0.02}{#1}}
\newcommand{\SpecialCharTok}[1]{\textcolor[rgb]{0.00,0.00,0.00}{#1}}
\newcommand{\StringTok}[1]{\textcolor[rgb]{0.31,0.60,0.02}{#1}}
\newcommand{\VerbatimStringTok}[1]{\textcolor[rgb]{0.31,0.60,0.02}{#1}}
\newcommand{\SpecialStringTok}[1]{\textcolor[rgb]{0.31,0.60,0.02}{#1}}
\newcommand{\ImportTok}[1]{#1}
\newcommand{\CommentTok}[1]{\textcolor[rgb]{0.56,0.35,0.01}{\textit{#1}}}
\newcommand{\DocumentationTok}[1]{\textcolor[rgb]{0.56,0.35,0.01}{\textbf{\textit{#1}}}}
\newcommand{\AnnotationTok}[1]{\textcolor[rgb]{0.56,0.35,0.01}{\textbf{\textit{#1}}}}
\newcommand{\CommentVarTok}[1]{\textcolor[rgb]{0.56,0.35,0.01}{\textbf{\textit{#1}}}}
\newcommand{\OtherTok}[1]{\textcolor[rgb]{0.56,0.35,0.01}{#1}}
\newcommand{\FunctionTok}[1]{\textcolor[rgb]{0.00,0.00,0.00}{#1}}
\newcommand{\VariableTok}[1]{\textcolor[rgb]{0.00,0.00,0.00}{#1}}
\newcommand{\ControlFlowTok}[1]{\textcolor[rgb]{0.13,0.29,0.53}{\textbf{#1}}}
\newcommand{\OperatorTok}[1]{\textcolor[rgb]{0.81,0.36,0.00}{\textbf{#1}}}
\newcommand{\BuiltInTok}[1]{#1}
\newcommand{\ExtensionTok}[1]{#1}
\newcommand{\PreprocessorTok}[1]{\textcolor[rgb]{0.56,0.35,0.01}{\textit{#1}}}
\newcommand{\AttributeTok}[1]{\textcolor[rgb]{0.77,0.63,0.00}{#1}}
\newcommand{\RegionMarkerTok}[1]{#1}
\newcommand{\InformationTok}[1]{\textcolor[rgb]{0.56,0.35,0.01}{\textbf{\textit{#1}}}}
\newcommand{\WarningTok}[1]{\textcolor[rgb]{0.56,0.35,0.01}{\textbf{\textit{#1}}}}
\newcommand{\AlertTok}[1]{\textcolor[rgb]{0.94,0.16,0.16}{#1}}
\newcommand{\ErrorTok}[1]{\textcolor[rgb]{0.64,0.00,0.00}{\textbf{#1}}}
\newcommand{\NormalTok}[1]{#1}
\usepackage{graphicx,grffile}
\makeatletter
\def\maxwidth{\ifdim\Gin@nat@width>\linewidth\linewidth\else\Gin@nat@width\fi}
\def\maxheight{\ifdim\Gin@nat@height>\textheight\textheight\else\Gin@nat@height\fi}
\makeatother
% Scale images if necessary, so that they will not overflow the page
% margins by default, and it is still possible to overwrite the defaults
% using explicit options in \includegraphics[width, height, ...]{}
\setkeys{Gin}{width=\maxwidth,height=\maxheight,keepaspectratio}
\IfFileExists{parskip.sty}{%
\usepackage{parskip}
}{% else
\setlength{\parindent}{0pt}
\setlength{\parskip}{6pt plus 2pt minus 1pt}
}
\setlength{\emergencystretch}{3em}  % prevent overfull lines
\providecommand{\tightlist}{%
  \setlength{\itemsep}{0pt}\setlength{\parskip}{0pt}}
\setcounter{secnumdepth}{0}
% Redefines (sub)paragraphs to behave more like sections
\ifx\paragraph\undefined\else
\let\oldparagraph\paragraph
\renewcommand{\paragraph}[1]{\oldparagraph{#1}\mbox{}}
\fi
\ifx\subparagraph\undefined\else
\let\oldsubparagraph\subparagraph
\renewcommand{\subparagraph}[1]{\oldsubparagraph{#1}\mbox{}}
\fi

%%% Use protect on footnotes to avoid problems with footnotes in titles
\let\rmarkdownfootnote\footnote%
\def\footnote{\protect\rmarkdownfootnote}

%%% Change title format to be more compact
\usepackage{titling}

% Create subtitle command for use in maketitle
\newcommand{\subtitle}[1]{
  \posttitle{
    \begin{center}\large#1\end{center}
    }
}

\setlength{\droptitle}{-2em}
  \title{NB}
  \pretitle{\vspace{\droptitle}\centering\huge}
  \posttitle{\par}
  \author{MENG}
  \preauthor{\centering\large\emph}
  \postauthor{\par}
  \predate{\centering\large\emph}
  \postdate{\par}
  \date{2018楠\textless{}9e\textgreater{}2閺\textless{}88\textgreater{}\textless{}88\textgreater{}13閺\textless{}83\textgreater{}}


\begin{document}
\maketitle

\subsubsection{\texorpdfstring{step1 download dataset
{[}link{]}(\url{http://www.dt.fee.unicamp.br/~tiago/smsspamcollection/})}{step1 download dataset {[}link{]}(http://www.dt.fee.unicamp.br/\textasciitilde{}tiago/smsspamcollection/)}}\label{step1-download-dataset-linkhttpwww.dt.fee.unicamp.brtiagosmsspamcollection}

\begin{Shaded}
\begin{Highlighting}[]
\KeywordTok{setwd}\NormalTok{(}\StringTok{'d:/rworkspace/ml'}\NormalTok{)}
\end{Highlighting}
\end{Shaded}

\subsubsection{step2 探索和准备数据}\label{step2-}

读取数据到sms\_raw中

\begin{Shaded}
\begin{Highlighting}[]
\KeywordTok{library}\NormalTok{(readr)}

\NormalTok{sms_raw <-}\StringTok{ }\KeywordTok{read_delim}\NormalTok{(}\StringTok{"D:/rworkspace/ml/SMSSpamCollection.txt"}\NormalTok{, }\StringTok{"}\CharTok{\textbackslash{}t}\StringTok{"}\NormalTok{, }\DataTypeTok{escape_double =} \OtherTok{FALSE}\NormalTok{, }\DataTypeTok{trim_ws =} \OtherTok{TRUE}\NormalTok{)}
\end{Highlighting}
\end{Shaded}

\begin{verbatim}
## Parsed with column specification:
## cols(
##   type = col_character(),
##   text = col_character()
## )
\end{verbatim}

\begin{Shaded}
\begin{Highlighting}[]
\KeywordTok{str}\NormalTok{(sms_raw)}
\end{Highlighting}
\end{Shaded}

\begin{verbatim}
## Classes 'tbl_df', 'tbl' and 'data.frame':    5572 obs. of  2 variables:
##  $ type: chr  "ham" "ham" "spam" "ham" ...
##  $ text: chr  "Go until jurong point, crazy.. Available only in bugis n great world la e buffet... Cine there got amore wat..." "Ok lar... Joking wif u oni..." "Free entry in 2 a wkly comp to win FA Cup final tkts 21st May 2005. Text FA to 87121 to receive entry question("| __truncated__ "U dun say so early hor... U c already then say..." ...
##  - attr(*, "spec")=List of 2
##   ..$ cols   :List of 2
##   .. ..$ type: list()
##   .. .. ..- attr(*, "class")= chr  "collector_character" "collector"
##   .. ..$ text: list()
##   .. .. ..- attr(*, "class")= chr  "collector_character" "collector"
##   ..$ default: list()
##   .. ..- attr(*, "class")= chr  "collector_guess" "collector"
##   ..- attr(*, "class")= chr "col_spec"
\end{verbatim}

将sms\_raw的类型从字符型变量改成因子型变量

\begin{Shaded}
\begin{Highlighting}[]
\NormalTok{sms_raw}\OperatorTok{$}\NormalTok{type <-}\StringTok{ }\KeywordTok{factor}\NormalTok{(sms_raw}\OperatorTok{$}\NormalTok{type)}
\KeywordTok{str}\NormalTok{(sms_raw}\OperatorTok{$}\NormalTok{type)}
\end{Highlighting}
\end{Shaded}

\begin{verbatim}
##  Factor w/ 2 levels "ham","spam": 1 1 2 1 1 2 1 1 2 2 ...
\end{verbatim}

用table查看type的数据,spam占比约13.4\%

\begin{Shaded}
\begin{Highlighting}[]
\KeywordTok{table}\NormalTok{(sms_raw}\OperatorTok{$}\NormalTok{type)}
\end{Highlighting}
\end{Shaded}

\begin{verbatim}
## 
##  ham spam 
## 4825  747
\end{verbatim}

\paragraph{处理和分析文本数据}

Corpus():创建了一个R对象来存储文本文档
VectorsSource():指示Corpus()函数使用向量sms\_raw\$text

\begin{Shaded}
\begin{Highlighting}[]
\KeywordTok{library}\NormalTok{(tm)}
\end{Highlighting}
\end{Shaded}

\begin{verbatim}
## Loading required package: NLP
\end{verbatim}

\begin{Shaded}
\begin{Highlighting}[]
\NormalTok{sms_corpus <-}\StringTok{ }\KeywordTok{Corpus}\NormalTok{(}\KeywordTok{VectorSource}\NormalTok{(sms_raw}\OperatorTok{$}\NormalTok{text))}
\KeywordTok{print}\NormalTok{(sms_corpus)}
\end{Highlighting}
\end{Shaded}

\begin{verbatim}
## <<SimpleCorpus>>
## Metadata:  corpus specific: 1, document level (indexed): 0
## Content:  documents: 5572
\end{verbatim}

inspect():查看语料库的内容

\begin{Shaded}
\begin{Highlighting}[]
\KeywordTok{inspect}\NormalTok{(sms_corpus[}\DecValTok{1}\OperatorTok{:}\DecValTok{3}\NormalTok{])}
\end{Highlighting}
\end{Shaded}

\begin{verbatim}
## <<SimpleCorpus>>
## Metadata:  corpus specific: 1, document level (indexed): 0
## Content:  documents: 3
## 
## [1] Go until jurong point, crazy.. Available only in bugis n great world la e buffet... Cine there got amore wat...                                            
## [2] Ok lar... Joking wif u oni...                                                                                                                              
## [3] Free entry in 2 a wkly comp to win FA Cup final tkts 21st May 2005. Text FA to 87121 to receive entry question(std txt rate)T&C's apply 08452810075over18's
\end{verbatim}

change upper letter to lower letter remove numbers remove stopwors()
remove punctuation remove space between 2 words

\begin{Shaded}
\begin{Highlighting}[]
\NormalTok{corpus_clean =}\StringTok{ }\KeywordTok{tm_map}\NormalTok{(sms_corpus, tolower)}
\NormalTok{corpus_clean =}\StringTok{ }\KeywordTok{tm_map}\NormalTok{(corpus_clean, removeNumbers)}
\NormalTok{corpus_clean =}\StringTok{ }\KeywordTok{tm_map}\NormalTok{(corpus_clean, removeWords, }\KeywordTok{stopwords}\NormalTok{())}
\NormalTok{corpus_clean =}\StringTok{ }\KeywordTok{tm_map}\NormalTok{(corpus_clean, removePunctuation)}
\NormalTok{corpus_clean =}\StringTok{ }\KeywordTok{tm_map}\NormalTok{(corpus_clean, stripWhitespace)}
\KeywordTok{inspect}\NormalTok{(corpus_clean[}\DecValTok{1}\OperatorTok{:}\DecValTok{3}\NormalTok{])}
\end{Highlighting}
\end{Shaded}

\begin{verbatim}
## <<SimpleCorpus>>
## Metadata:  corpus specific: 1, document level (indexed): 0
## Content:  documents: 3
## 
## [1] go jurong point crazy available bugis n great world la e buffet cine got amore wat                     
## [2] ok lar joking wif u oni                                                                                
## [3] free entry wkly comp win fa cup final tkts st may text fa receive entry questionstd txt ratetcs apply s
\end{verbatim}

通过\textbf{标记化}过程将消息分解成单个单词组成的组
一个记号(token)就是一个文本字符串的单个元素,本例中token就是单词
DocumentTermMatrix():
将一个语料库作为输入,并创建一个\textbf{稀疏矩阵}的数据结构

\begin{Shaded}
\begin{Highlighting}[]
\CommentTok{# 创建稀疏矩阵}
\NormalTok{sms_dtm =}\StringTok{ }\KeywordTok{DocumentTermMatrix}\NormalTok{(corpus_clean)}
\end{Highlighting}
\end{Shaded}

\begin{enumerate}
\def\labelenumi{\arabic{enumi}.}
\tightlist
\item
  建立测试和训练数据集
\end{enumerate}

\begin{Shaded}
\begin{Highlighting}[]
\NormalTok{sms_raw_train <-}\StringTok{ }\NormalTok{sms_raw[}\DecValTok{1}\OperatorTok{:}\DecValTok{4825}\NormalTok{, ]}
\NormalTok{sms_raw_test <-}\StringTok{ }\NormalTok{sms_raw[}\DecValTok{4826}\OperatorTok{:}\DecValTok{5572}\NormalTok{, ]}
\end{Highlighting}
\end{Shaded}

输入文档-单词矩阵

\begin{Shaded}
\begin{Highlighting}[]
\NormalTok{sms_dtm_train <-}\StringTok{ }\NormalTok{sms_dtm[}\DecValTok{1}\OperatorTok{:}\DecValTok{4825}\NormalTok{, ]}
\NormalTok{sms_dtm_test <-}\StringTok{ }\NormalTok{sms_dtm[}\DecValTok{4826}\OperatorTok{:}\DecValTok{5572}\NormalTok{, ]}
\end{Highlighting}
\end{Shaded}

最终语料库

\begin{Shaded}
\begin{Highlighting}[]
\NormalTok{sms_corpus_train <-}\StringTok{ }\NormalTok{corpus_clean[}\DecValTok{1}\OperatorTok{:}\DecValTok{4825}\NormalTok{ ]}
\NormalTok{sms_corpus_test <-}\StringTok{ }\NormalTok{corpus_clean[}\DecValTok{4826}\OperatorTok{:}\DecValTok{5572}\NormalTok{ ]}
\end{Highlighting}
\end{Shaded}

ham和spam在test和train中占比差不多,说明分配比较均匀

\begin{Shaded}
\begin{Highlighting}[]
\KeywordTok{prop.table}\NormalTok{(}\KeywordTok{table}\NormalTok{(sms_raw_train}\OperatorTok{$}\NormalTok{type))}
\end{Highlighting}
\end{Shaded}

\begin{verbatim}
## 
##       ham      spam 
## 0.8654922 0.1345078
\end{verbatim}

\begin{Shaded}
\begin{Highlighting}[]
\KeywordTok{prop.table}\NormalTok{(}\KeywordTok{table}\NormalTok{(sms_raw_test}\OperatorTok{$}\NormalTok{type))}
\end{Highlighting}
\end{Shaded}

\begin{verbatim}
## 
##       ham      spam 
## 0.8688086 0.1311914
\end{verbatim}

\begin{enumerate}
\def\labelenumi{\arabic{enumi}.}
\setcounter{enumi}{1}
\tightlist
\item
  可视化文本数据-词云 random.order=F: 顺序排列,词频越高,越接近中心
  min.freq=40: 在至少40条短信中出现过
\end{enumerate}

\begin{Shaded}
\begin{Highlighting}[]
\KeywordTok{library}\NormalTok{(}\StringTok{'wordcloud'}\NormalTok{)}
\end{Highlighting}
\end{Shaded}

\begin{verbatim}
## Loading required package: RColorBrewer
\end{verbatim}

\begin{Shaded}
\begin{Highlighting}[]
\KeywordTok{wordcloud}\NormalTok{(sms_corpus_train, }\DataTypeTok{min.freq =} \DecValTok{40}\NormalTok{, }\DataTypeTok{random.order =}\NormalTok{ F)}
\end{Highlighting}
\end{Shaded}

\includegraphics{NB_files/figure-latex/unnamed-chunk-13-1.pdf}

在训练集中,对标签为spam和ham分别做图,进行比较。spam中,比较多的是free,prize,推测是垃圾信息

\begin{Shaded}
\begin{Highlighting}[]
\NormalTok{spam =}\StringTok{ }\KeywordTok{subset}\NormalTok{(sms_raw_train, type }\OperatorTok{==}\StringTok{ 'spam'}\NormalTok{)}
\NormalTok{ham =}\StringTok{ }\KeywordTok{subset}\NormalTok{(sms_raw_train, type }\OperatorTok{==}\StringTok{ 'ham'}\NormalTok{)}
\KeywordTok{wordcloud}\NormalTok{(spam}\OperatorTok{$}\NormalTok{text, }\DataTypeTok{max.words =} \DecValTok{40}\NormalTok{, }\DataTypeTok{scale =} \KeywordTok{c}\NormalTok{(}\DecValTok{3}\NormalTok{, }\FloatTok{0.5}\NormalTok{))}
\end{Highlighting}
\end{Shaded}

\begin{verbatim}
## Warning in strwidth(words[i], cex = size[i], ...): 'mbcsToSbcs'里转换'鎷
## �2000'出错:<e6>代替了dot
\end{verbatim}

\begin{verbatim}
## Warning in strwidth(words[i], cex = size[i], ...): 'mbcsToSbcs'里转换'鎷
## �2000'出错:<8b>代替了dot
\end{verbatim}

\begin{verbatim}
## Warning in strwidth(words[i], cex = size[i], ...): 'mbcsToSbcs'里转换'鎷
## �2000'出错:<a2>代替了dot
\end{verbatim}

\begin{verbatim}
## Warning in text.default(x1, y1, words[i], cex = size[i], offset = 0, srt =
## rotWord * : 'mbcsToSbcs'里转换'鎷�2000'出错:<e6>代替了dot
\end{verbatim}

\begin{verbatim}
## Warning in text.default(x1, y1, words[i], cex = size[i], offset = 0, srt =
## rotWord * : 'mbcsToSbcs'里转换'鎷�2000'出错:<8b>代替了dot
\end{verbatim}

\begin{verbatim}
## Warning in text.default(x1, y1, words[i], cex = size[i], offset = 0, srt =
## rotWord * : 'mbcsToSbcs'里转换'鎷�2000'出错:<a2>代替了dot
\end{verbatim}

\begin{verbatim}
## Warning in text.default(x1, y1, words[i], cex = size[i], offset = 0, srt =
## rotWord * : Unicode字符U+62e2不带字体度量
\end{verbatim}

\begin{verbatim}
## Warning in strwidth(words[i], cex = size[i], ...): 'mbcsToSbcs'里转换'鎷
## �1000'出错:<e6>代替了dot
\end{verbatim}

\begin{verbatim}
## Warning in strwidth(words[i], cex = size[i], ...): 'mbcsToSbcs'里转换'鎷
## �1000'出错:<8b>代替了dot
\end{verbatim}

\begin{verbatim}
## Warning in strwidth(words[i], cex = size[i], ...): 'mbcsToSbcs'里转换'鎷
## �1000'出错:<a2>代替了dot
\end{verbatim}

\begin{verbatim}
## Warning in text.default(x1, y1, words[i], cex = size[i], offset = 0, srt =
## rotWord * : 'mbcsToSbcs'里转换'鎷�1000'出错:<e6>代替了dot
\end{verbatim}

\begin{verbatim}
## Warning in text.default(x1, y1, words[i], cex = size[i], offset = 0, srt =
## rotWord * : 'mbcsToSbcs'里转换'鎷�1000'出错:<8b>代替了dot
\end{verbatim}

\begin{verbatim}
## Warning in text.default(x1, y1, words[i], cex = size[i], offset = 0, srt =
## rotWord * : 'mbcsToSbcs'里转换'鎷�1000'出错:<a2>代替了dot
\end{verbatim}

\begin{verbatim}
## Warning in text.default(x1, y1, words[i], cex = size[i], offset = 0, srt =
## rotWord * : Unicode字符U+62e2不带字体度量
\end{verbatim}

\includegraphics{NB_files/figure-latex/unnamed-chunk-14-1.pdf}

\begin{Shaded}
\begin{Highlighting}[]
\KeywordTok{wordcloud}\NormalTok{(ham}\OperatorTok{$}\NormalTok{text, }\DataTypeTok{max.words =} \DecValTok{40}\NormalTok{, }\DataTypeTok{scale =} \KeywordTok{c}\NormalTok{(}\DecValTok{3}\NormalTok{, }\FloatTok{0.5}\NormalTok{))}
\end{Highlighting}
\end{Shaded}

\includegraphics{NB_files/figure-latex/unnamed-chunk-14-2.pdf} 3.
为频繁出现的单词创建指示特征 findFreqTerms():
输入\textbf{文档-单词矩阵}, 返回一个字符向量

\begin{Shaded}
\begin{Highlighting}[]
\CommentTok{# 参数5表示该向量中断的单词在矩阵中至少出现5次,类型是character}
\NormalTok{sms_dict =}\StringTok{ }\KeywordTok{findFreqTerms}\NormalTok{(sms_dtm_train, }\DecValTok{5}\NormalTok{)}

\CommentTok{#把数据集中所有频率超过5次的单词保存成新的数据集}
\NormalTok{sms_train =}\StringTok{ }\KeywordTok{DocumentTermMatrix}\NormalTok{(sms_corpus_train, }\KeywordTok{list}\NormalTok{(}\DataTypeTok{dictionary =}\NormalTok{ sms_dict))}
\NormalTok{sms_test =}\StringTok{ }\KeywordTok{DocumentTermMatrix}\NormalTok{(sms_corpus_test, }\KeywordTok{list}\NormalTok{(}\DataTypeTok{dictionary =}\NormalTok{ sms_dict))}
\end{Highlighting}
\end{Shaded}

朴素贝叶斯分类器通常是训练具有明确特征的数据。
因为稀疏矩阵中的元素表示一个单词出现的次数,我们需要将其改变成因子变量,根据单词是否出现,简单的表示为yes或者no

\begin{Shaded}
\begin{Highlighting}[]
\CommentTok{# 自定义函数convert_counts()}
\NormalTok{convert_counts =}\StringTok{ }\ControlFlowTok{function}\NormalTok{(x)\{ }
\NormalTok{  x =}\StringTok{ }\KeywordTok{ifelse}\NormalTok{(x}\OperatorTok{>}\DecValTok{0}\NormalTok{, }\DecValTok{1}\NormalTok{, }\DecValTok{0}\NormalTok{)  }\CommentTok{#x > 0, 用1替换,否则用0替换}
\NormalTok{  x =}\StringTok{ }\KeywordTok{factor}\NormalTok{(x, }\DataTypeTok{levels =} \KeywordTok{c}\NormalTok{(}\DecValTok{0}\NormalTok{, }\DecValTok{1}\NormalTok{), }\DataTypeTok{labels =} \KeywordTok{c}\NormalTok{(}\StringTok{'"No"'}\NormalTok{, }\StringTok{'"Yes"'}\NormalTok{))}
  \KeywordTok{return}\NormalTok{(x)}
\NormalTok{  \}}
\end{Highlighting}
\end{Shaded}

将convert\_counts()应用到每列稀疏矩阵。
apply()来实现,而非for或者while循环

\begin{Shaded}
\begin{Highlighting}[]
\CommentTok{#MARGIN = 1 按行,MARGIN = 2按列}
\NormalTok{sms_train <-}\StringTok{ }\KeywordTok{apply}\NormalTok{(sms_train, }\DataTypeTok{MARGIN =} \DecValTok{2}\NormalTok{, convert_counts)}
\NormalTok{sms_test <-}\StringTok{ }\KeywordTok{apply}\NormalTok{(sms_test, }\DataTypeTok{MARGIN =} \DecValTok{2}\NormalTok{, convert_counts)}
\end{Highlighting}
\end{Shaded}

\subsubsection{step3 基于数据训练模型}\label{step3-}

\begin{Shaded}
\begin{Highlighting}[]
\KeywordTok{library}\NormalTok{(e1071)}
\NormalTok{sms_classifier =}\StringTok{ }\KeywordTok{naiveBayes}\NormalTok{(sms_train, sms_raw_train}\OperatorTok{$}\NormalTok{type)}
\CommentTok{#class(sms_train)}
\CommentTok{#class(sms_classifier)}
\CommentTok{#class(sms_raw_train$type)}
\end{Highlighting}
\end{Shaded}

\subsubsection{step4 评估模型的性能}\label{step4-}

\begin{Shaded}
\begin{Highlighting}[]
\NormalTok{sms_test_pred =}\StringTok{ }\KeywordTok{predict}\NormalTok{(sms_classifier, sms_test)}
\KeywordTok{class}\NormalTok{(sms_test_pred)}
\end{Highlighting}
\end{Shaded}

\begin{verbatim}
## [1] "factor"
\end{verbatim}

\begin{Shaded}
\begin{Highlighting}[]
\KeywordTok{library}\NormalTok{(gmodels)}
\KeywordTok{CrossTable}\NormalTok{(sms_test_pred, sms_raw_test}\OperatorTok{$}\NormalTok{type,}\DataTypeTok{prop.chisq =}\NormalTok{ F, }\DataTypeTok{prop.t =}\NormalTok{ F, }\DataTypeTok{dnn =} \KeywordTok{c}\NormalTok{(}\StringTok{'predicted'}\NormalTok{, }\StringTok{'actual'}\NormalTok{))}
\end{Highlighting}
\end{Shaded}

\begin{verbatim}
## 
##  
##    Cell Contents
## |-------------------------|
## |                       N |
## |           N / Row Total |
## |           N / Col Total |
## |-------------------------|
## 
##  
## Total Observations in Table:  747 
## 
##  
##              | actual 
##    predicted |       ham |      spam | Row Total | 
## -------------|-----------|-----------|-----------|
##          ham |       647 |        12 |       659 | 
##              |     0.982 |     0.018 |     0.882 | 
##              |     0.997 |     0.122 |           | 
## -------------|-----------|-----------|-----------|
##         spam |         2 |        86 |        88 | 
##              |     0.023 |     0.977 |     0.118 | 
##              |     0.003 |     0.878 |           | 
## -------------|-----------|-----------|-----------|
## Column Total |       649 |        98 |       747 | 
##              |     0.869 |     0.131 |           | 
## -------------|-----------|-----------|-----------|
## 
## 
\end{verbatim}

\subsubsection{step5 提升模型性能}\label{step5-}

\begin{Shaded}
\begin{Highlighting}[]
\NormalTok{sms_classifier2 =}\StringTok{ }\KeywordTok{naiveBayes}\NormalTok{(sms_train, sms_raw_train}\OperatorTok{$}\NormalTok{type, }\DataTypeTok{laplace =} \DecValTok{1}\NormalTok{)}
\NormalTok{sms_test_pred2 =}\StringTok{ }\KeywordTok{predict}\NormalTok{(sms_classifier2, sms_test)}
\KeywordTok{CrossTable}\NormalTok{(sms_test_pred2, sms_raw_test}\OperatorTok{$}\NormalTok{type, }\DataTypeTok{prop.chisq =}\NormalTok{ F, }\DataTypeTok{prop.t =}\NormalTok{ F, }\DataTypeTok{prop.r=}\NormalTok{F, }\DataTypeTok{dnn =} \KeywordTok{c}\NormalTok{(}\StringTok{'predicted'}\NormalTok{,}\StringTok{'test'}\NormalTok{))}
\end{Highlighting}
\end{Shaded}

\begin{verbatim}
## 
##  
##    Cell Contents
## |-------------------------|
## |                       N |
## |           N / Col Total |
## |-------------------------|
## 
##  
## Total Observations in Table:  747 
## 
##  
##              | test 
##    predicted |       ham |      spam | Row Total | 
## -------------|-----------|-----------|-----------|
##          ham |       647 |        17 |       664 | 
##              |     0.997 |     0.173 |           | 
## -------------|-----------|-----------|-----------|
##         spam |         2 |        81 |        83 | 
##              |     0.003 |     0.827 |           | 
## -------------|-----------|-----------|-----------|
## Column Total |       649 |        98 |       747 | 
##              |     0.869 |     0.131 |           | 
## -------------|-----------|-----------|-----------|
## 
## 
\end{verbatim}


\end{document}
